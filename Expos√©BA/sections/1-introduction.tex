\section{Einleitung}
\label{sec:introduction}
Die Veröffentlichung des großen Sprachmodells GPT3 des Unternehmens OpenAI wurde in weiten Teilen der Öffentlichkeit als Revolution im Bereich der Künstlichen Intelligenz (KI) aufgefasst. Ein großes Sprachmodell ist ein maschinelles Lernmodell, das darauf trainiert ist, ein allgemeines Sprachverständnis aufzubauen um damit menschliche Texte zu verstehen und menschenähnliche Texte zu generieren. Es ist groß in Bezug auf die Menge der Daten, mit denen es trainiert wurde, und die Komplexität der Aufgaben, die es ausführen kann. Sprachmodelle sind eine Form der generativen KI, das heißt sie generieren neue Texte auf Basis von vorhandenen Lerndaten und spezifischen Eingaben. ChatGPT ist eine Chatbot-Implementierung auf Basis des GPT-Modells, die speziell für den Einsatz in Dialogen und Konversationen mit Menschen entwickelt wurde. ChatGPT funktioniert sehr gut, um im Rahmen der Konversation mit einem ChatBot Informationen über vorhandenes Weltwissen zu erlangen, wobei der ChatBot durchaus auch in der Lage ist vorhandene Informationen kreativ zu verarbeiten. Der große Vorteil dabei ist, dass sowohl Anfragen als auch Instruktionen in natürlicher Sprache erfolgen können, sodass jederman in der Lage ist mit der Künstlichen Intelligenz zu interagieren. Das Weltwissen des GPT3-Modells von OpenAI basiert beispielsweise auf circa 175 Milliarden Parametern, welche das Modell aus den Trainingsdaten extrahiert hat. GPT3 hat keinen Zugang zu Informationen außerhalb der Daten, mit denen es trainiert wurde. Es kann keine neuen Informationen aus dem Internet oder anderen Quellen abrufen, nachdem es trainiert wurde. Das bedeutet, das man den Chatbot sehr gut nach den Mitgliedern der Beatles und deren Geburtstag befragen kann, aber bei Fragen zu spezifischem Domänenwissen keine zufriedenstellende Antwort erhalten wird. \parencite{noauthor_chatgpt_2023} \\

Im Rahmen dieser Arbeit steht als spezifisches Domänenwissen das Wissen über verfahrenstechnische Anlagen im Fokus. Im Zuge des Design- und Engineeringprozesses solcher Anlagen entsteht eine Vielzahl technischer Daten, Spezifikationen und Beschreibungen. Diese Informationen werden mit Hilfe systematischer Softwareunterstützung häufig von externen Planern erzeugt und zum sogenannten Handover in unterschiedlicher Form an den Anlagenbetreiber übergeben. In der chemischen und pharmazeutischen Industrie erstreckt sich die Lebensdauer von Produktionsanlagen üblicherweise über mehrere Jahrzehnte hinweg, wobei 30 bis 40 Jahre Betriebszeit keine Seltenheit sind. Die Planungs- und Bauphase einer Anlage beträgt circa drei bis fünf Jahre. Historisch gesehen lag der Fokus der systematischen Softwareunterstützung hauptsächlich auf der kürzeren Phase des Engineerings und Designs. Als nächster Schritt in der Innovation steht eine ganzheitliche Betrachtung des gesamten Anlagenlebenszyklus im Fokus. Dabei reicht die Perspektive von den anfänglichen Designphasen bis hin zum gesamten Betriebslebenszyklus. Anlagenbetreiber streben vermehrt nach Wegen, den wirtschaftlichen Ertrag ihrer Investitionen zu maximieren. Dabei leiden gerade Kapazitätserweiterungsprojekte oft unter unzureichender Anlagendokumentation. Die Übergabe von den Phasen Projektentwurf und Bau zum Betrieb erfolgt häufig auf der Basis minimaler gemeinsamer Dokumentationsformate, wie einfache Zeichenformate, Excel-basierte Berichte oder PDF-Dokumente, die wenig oder keine semantischen Daten enthalten, welche wertvoll  genutzt werden können. \parencite{wiedau_asset_2018} \\

Das Ziel von DEXPI (Data Exchange in the Process Industry) ist die Entwicklung und Förderung eines gemeinsamen Datenaustauschstandards für die Prozessindustrie, der alle Phasen des Lebenszyklus einer Prozessanlage abdeckt, von der Spezifikation funktionaler Anforderungen bis hin zu den im Einsatz befindlichen Anlagen. Der aktuelle Schwerpunkt des DEXPI Vereins liegt auf dem Austausch von Rohrleitungs- und Instrumentenfließbilder. \parencite{noauthor_dexpi_2023} \\



