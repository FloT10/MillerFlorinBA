\section{Problemstellung und Zielsetzung}
\label{sec:related_work}
Im Lichte der Inovationen im Bereich der großen Sprachmodelle erscheint die Enticklung eines standardisierten objektorientierten Datenmodells für verfahrenstechnische Anlagen als optimale Vorraussetzung, um diese Daten maschienell verarbeiten und abrufen zu können. Aus Sicht des Anlagenbetreibers wäre es über den gesamten Anlagenbetrieb hinweg hilfreich, einen schnellen und niedrigschwelligen Zugang zu den jeweils aktuellen Anlageninformationen zu erhalten.

Im Rahmen dieser Arbeit soll untersucht werden welche Ansätze es zur Anwendung von großen Sprachmodellen im Rahmen der Anlagendokumentation von verfahrenstechnischen Anlagen gibt.
Dabei soll zu den Themen große Sprachmodelle und  Anlagendokumentation von verfahrenstechnischen Anlagen jeweils der aktuelle Stand der Technik umrissen werden.
Es soll auch ein eigener Ansatz verfolgt werden.

Zielsetzung das Ansatzes muss es sein, aus Sicht eines Anlagenbetreibers, einen komfortablen Zugang zur Abfrage von Equipment- und später Anlageninformationen zu erhalten. Komfortabel wäre es für den Anlagenbetreiber wenn er die Informationen über die Anlagen mit einem definierten Mindeststandard über ein interoperables Datenformat vom Planer erhalten würde. Damit könnte er auf die semantischen Daten direkt zugreifen und diese in unterschiedlicher Weise nutzen. 

Das zentrale Austauschdokument für Verfahrenstechnische Anlagen ist das Rohrleitungs- und Instrumentenfließbild. Es bildet alle relevanten Anlagenbestandteile (z.B. Tanks, Wärmetauscher, Rohrleitungen, Pumpen, Mess-, Steuerungs- und Regelungstechnik usw.) ab. Zusammen mit den Informationen über die einzelnen Equipments (z.B. Volumen, Nenndurchmesser, Förderleistung usw.) und einer Anlagenbeschreibung wird die grundlegende Funktionsweise einer Anlage nachvollziehbar. 

Große Sprachmodelle könnten dabei sowohl hinsichtlich des Parsens der interoperablen Dateien zur Anwendung kommen, als auch als Werkzeug um mit diesen Daten in Interaktion zu treten.