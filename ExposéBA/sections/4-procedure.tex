\section{Methodik und Vorgehen}
\label{sec:procedure}

Um die Forschungsfrage, wie die Anwendung großer Sprachmodelle in der Anlagendokumentation von verfahrenstechnischen Anlagen für Anlagenbetreiber sinnvoll sein kann, zu beantworten, wird folgende Methodik verfolgt.

\subsection{Forschungsdesign}
Das Forschungsdesign wird adaptiv und iterativ ausgelegt. Es wird erwartet, dass der Forschungsplan im Verlauf der Arbeit dynamisch bleibt, um auf neue Erkenntnisse und Anforderungen zu reagieren. Der klare Plan am Anfang dient als Leitfaden, jedoch werden Anpassungen und Modifikationen in Absprache mit dem Betreuer vorgenommen.

\subsection{Literaturrecherche}
Eine gründliche Literaturrecherche zu bestehenden Arbeiten über den Einsatz von Sprachmodellen in technischen Dokumentationen sowie spezifisch in der verfahrenstechnischen Industrie wird durchgeführt. Dies bildet die Grundlage für das Verständnis des aktuellen Standes, bestehender Probleme und erfolgreicher Anwendungsgebiete.

\subsection{Datenerhebung und Analyse}
Nach Möglichkeit wird eine Datenerhebung durch Interviews mit Fachleuten aus der verfahrenstechnischen Industrie und Anlagenbetreibern erfolgen. Diese Interviews dienen der Ermittlung von Erfahrungen, Bedenken und potenziellen Einsatzgebieten von Sprachmodellen in der Anlagendokumentation. Die analysierten Daten helfen, Schlussfolgerungen und Empfehlungen abzuleiten.

\subsection{Experimente und Modellierung}
Es könnten Experimente durchgeführt werden, um die Machbarkeit und Leistungsfähigkeit von Sprachmodellen in der Anlagendokumentation zu testen. Hierbei werden verschiedene Modelle und Anwendungsszenarien betrachtet, um ihre Auswirkungen auf die Effizienz der Anlagendokumentation zu bewerten.

\subsection{Risiken und ethische Aspekte}
Es wird sorgfältig auf mögliche Risiken und ethische Bedenken im Rahmen dieser Forschung geachtet. Die Hauptpunkte werden klar benannt und Ansätze zur Lösung solcher Probleme werden erläutert.

Die angewandte Methodik ist darauf ausgerichtet, ein tiefgehendes Verständnis für die Anwendung großer Sprachmodelle in der Anlagendokumentation zu erlangen und praktische Einsichten für potenzielle Verbesserungen in diesem Bereich zu liefern.
